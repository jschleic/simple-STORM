\documentclass[DIV12,a4paper]{scrartcl}
\usepackage[ngerman]{babel}
\usepackage[T1]{fontenc}
\usepackage[utf8]{inputenc}
\usepackage{ae,aecompl}
\usepackage{hyperref}

\begin{document}
\title{Tool zur Auswertung von Daten in der Lokalisationsmikroskopie}
\subtitle{Version 0.3.0}
\author{Joachim Schleicher <J.Schleicher@stud.uni-heidelberg.de>}
\maketitle

\section{Ziel}
%STORM=?
Ziel ist die Analyse von Bilderstapeln, die mit STORM / PALM Mikroskopen 
aufgenommen wurden.
Die Aufnahmemethode ermöglicht es, Auflösungen bis ca. 20nm zu erreichen.
Um diese Auflösung in Aufnahmen mit niedriger Intensität zu bekommen, ist
eine sorgfältige Analyse der Daten erforderlich.
Das Programm \texttt{simplestorm} passt die Noise-Filterung automatisch an 
die gemesssenen Daten an. 

\section{Installation und erster Start des Programms}
Das Programm ist in der aktuellen Version ohne Installationsroutine sofort einsatzbereit.
Als Kommandozeilentool wird der Umgang mit einer Shell vorausgesetzt.
Um \texttt{simplestorm} systemweit ausführen zu können, ist es sinnvoll, 
das ausführbare Programm 
in einen Ordner zu legen, in dem es vom System gefunden werden kann.
Dazu wird der Ordner zu der Umgebungsvariablen \texttt{PATH} hinzugefügt.

\paragraph{Windows} 
Entpacken des Programms z.B. nach \texttt{C:\textbackslash simplestorm\textbackslash}, 
diesen Ordner unter \emph{System-> Erweitert->Umgebungsvariablen} zum \texttt{PATH} 
hinzufügen.

\paragraph{Linux} % z.b. /home/inst/bin/simplestorm + libvigraimpex.so
Möglicherweise muss das Programm zunächst aus den Quellen kompiliert werden.
Mehr dazu unter \ref{sec:Kompilieren}.
Den \texttt{PATH} kann man gewöhnlich in der Datei \texttt{.bashrc} editieren.

\paragraph{Start des Programms} 
Ein Aufruf von der Kommandozeile aus sieht gewöhnlich folgendermaßen aus
(vorher mit \texttt{cd} ins Datenverzeichnis wechseln):\\
\texttt{storm -\,-threshold=40 inputfilm.sif}\\
Der Fortschritt wird durch Ausgabe des aktuell bearbeiteten Frames angezeigt.
Die gefundenen Spots werden in einer Textdatei gespeichert, die standardmäßig
denselben Namen wie das Datenfile -- mit der Endung .txt -- trägt. %+Format
Um die Daten auch gleich betrachten zu können, wird zusätzlich ein Bild erzeugt, 
in dem die Spots (mit der Intensität im Film gewichtet) aufsummiert werden. 
Ein Parameter \texttt{threshold} muss manuell gewählt werden, um Signale 
aus dem Hintergrund auszuschließen und die tatsächlichen Floureszenz-Farbstoffe 
zu detektieren.

\section{Erweiterte Einstellungsmöglichkeiten}
Als Kommandozeilenoptionen stehen verschiedene Optionen zur Verfügung.
Diese können (sofern nicht anders gekennzeichnet) beliebig kombiniert werden.
Eine kurze Übersicht lässt sich mit \texttt{storm -\,-help} anzeigen.

\subsection{Vor-Filterung}
Um Rauschen in den Daten zu entfernen wird automatisch ein Filter erzeugt.
Statt dieses Filter aus den Daten zu generieren, kann auch eine \emph{tif}-Bilddatei 
geladen werden, mit der alle Bilder vor der Verarbeitung im Fourierraum 
multipliziert werden. 

\subsection{Vergrößerungsfaktor}
Die Spots werden auf einen Bruchteil eines Pixels lokalisiert. Dazu interpoliert 
das Programm die Daten mit Methoden aus der Bildverarbeitung. Je höher dieser 
Vergrößerungsfaktor, desto genauer können die Spots lokalisiert werden, desto 
länger dauert aber die Suche. Ein guter Kompromiss ist \texttt{factor=8}.
Hier könnte es sinnvoll sein, Zweierpotenzen zu bevorzugen, um Rundungsfehler 
zu minimieren.

\subsection{Bereichsauswahl}
Um die Verarbeitung zu beschleunigen (z.B. Vorschau), kann die Verarbeitung auf 
eine Untermenge der Frames beschränkt werden. Die Option \texttt{frames} erwartet
einen Bereich, der an die Python-Slicefunktion von Arrays angelehnt ist. 
Für einen Film mit 1000 Einzelbildern sind beispielsweise erlaubt:\\
\texttt{frames=200:500} \# analysiert die Frames 200 bis 499\\
\texttt{frames=1::4} \# startet mit Frame 1, analysiert nur jedes vierte Bild\\
\texttt{frames=100:-100} \# beginnt mit Frame 100 und analysiert bis Frame 900 (- zählt von der Gesamtzahl rückwärts) % oder 899?

\subsection{mögliche Erweiterungen}
mal sehen, was für Erweiterungen es demnächst geben wird...
% Ideen, was für zukünftige Releases erwartet werden kann...

\section{Kompilieren des Sourcecodes}
\label{sec:Kompilieren}
\subsection{Requirements}
Der Code verwendet intensiv Header und Bibliotheksfunktionen des Bildverarbeitungs-Pakets \emph{VIGRA}.
Notwendig ist mindestens Version 1.7.1; um auch ältere Andor-Sif-Dateien 
öffnen zu können, empfehlen wir die Development-Version. Hinweise zum Download 
von \emph{VIGRA} sind auf der Homepage \url{http://hci.iwr.uni-heidelberg.de/vigra/} zu finden.

\subsection{CMake, Einstellungen}
% HDF5 optional, 

\section{Lizenz}
Siehe Quellcode.

\end{document}
